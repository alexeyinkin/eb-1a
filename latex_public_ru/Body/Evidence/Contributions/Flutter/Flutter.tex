\subsubsection{Flutter Code Editor}
\label{subsubsec:ContributionsFlutter}

\SubSubSubSection{Контекст}

Flutter -- это технология Google для крос-платформенных приложений для веба, мобильных устройств и комьютеров.
По утверждению самого Google она позволила им сократить инженерные затраты на 70\%
для одного из их приложений в сравнении с отдельной разработкой под Android и iOS \ExhibitRef{FlutterGooglePay}.

`Flutter Code Editor' -- это программный продукт, разработанный в Akvelon под руководством Алексея Инкина.
Это редактор кода, который может использоваться как часть любого приложения на Flutter.
В нём уникальное сочетанием функционала и поддерживаемых платформ, как продемонстрировано ниже,
и поэтому он представляет собой оригинальный научный вклад в сферу разработки ПО.

Для контекста и визуального представления о редакторе см. \ExhibitRef{FceDemo}.


\SubSubSubSection{Доказательство, что Алексей Инкин -- главный автор редактора}

Ведущая роль Алексея Инкина указана в письме от \MrAkvelonT:

\ParagraphQuoteByExhibit{%
    <\dots>%
}{\MrAkvelonT}{LetterAkvelon}

На GitHub можно независимо проверить, что Алексей Инкин внёс почти в 5 раз больше коммитов
в код редактора, чем следующий автор \ExhibitRef{FceGitHubContributors}.

Наконец, Алексей Инкин указан первым в команде в анонсе
\ExhibitRef{AkvelonFceAnnouncement}.


\SubSubSubSection{Доказательство, что продукт -- второй по популярности редактор кода на Flutter}

Официальный репозиторий пакетов для Dart и Flutter -- `https://pub.dev'.
Он показывает метрику популярности для пакетов, но там нет категорий пакетов.

Чтобы прицельно отобрать редакторы кода для сравнения их популярности,
можно воспользоваться сайтом FlutterGems.dev.
Как сказано в их Твиттере, это

\ParagraphQuoteByExhibit{%
    собираемый вручную гайд по пакетам, где больше 5.5 тысяч самых полезных и популярных пакетов для Dart и Flutter
    сгруппированы по функционалу.%
}{Flutter Gems Twitter}{FlutterGemsTwitter}

Для контекста: Dart это язык программирования, который используется в технологии Flutter,
они часто встречаются вместе в этой петиции.

На FlutterGems.dev есть категория с пакетами для редактирования кода и подсветки синтаксиса, в ней 12 пакетов
\ExhibitRef{FlutterGemsEditors}.
Из них 7 -- в хорошем и среднем статусе поддержки.
Из них нужно исключить следующие:

\begin{itemize}

    \item flutter\_highlighter and highlighter -- не редакторы,
    они только подсвечивают статический нередактируемый код.

    \item json\_editor -- редактор для одного языка, JSON \ExhibitRef{JsonWikipedia},
    в то время как универсальные редакторы поддерживают код на более чем 100 языках.

\end{itemize}

Если отсортировать оставшиеся пакеты по лайкам на pub.dev, то получится:

\begin{center}
    \begin{tabular}{|l|l|r|r|}
        \hline
        & Название & Лайки & Популярность\\
        \hline
        1 & code\_text\_field & 135 & 93\%\\
        2 & code\_editor & 94 & 83\%\\
        3 & flutter\_code\_editor & 80 & 89\%\\
        4 & codemirror & 15 & 83\%\\
        \hline
    \end{tabular}
\end{center}

В этой таблице метрика популярности взята со страницы каждого пакета на pub.dev,
который является первоисточником всех статистик \ExhibitRef{CodeEditorPopularity}.

Как сказано на справочной странице pub.dev, лайки -- это то, что разработчики выставляют пакетам вручную,
в то время как популярность -- это

\ParagraphQuoteByExhibit{%
    процентиль от 100\% (топ 1\% самых используемых пакетов)
    до 0\% (наименее используемый пакет).%
}{Справочная страница Pub.dev}{PubScoring}

Таким образом, правильная метрика -- это популярность, а не лайки.
Это делает пакет flutter\_code\_editor вторым по популярности универсальным редактором кода для Flutter.

Интересно, что `codemirror', четвёртый пакет в списке, создан Google,
как указано на его странице на pub.dev в разделе `Publisher' \ExhibitRef{CodeEditorPopularity}.
Продукт Алексея Инкина в этой категории популярнее, чем конкурирующий продукт от самого Google.

Примечание: Процентили популярности 93\% и 89\% не должны восприниматься как низкие сами по себе.
Редакторы кода -- это нишевые пакеты для создания приложений для программистов.
Они ожидаемо менее популярные, чем красивые кнопки, списки, таймеры и другие элементы,
которые используются в приложениях для массового использования и имеют популярность 98--100\% на pub.dev.


\SubSubSubSection{Доказательство большой значимости продукта}

Из рекомендательного письма от Google:

\ParagraphQuoteByExhibit{%
    <\dots>%
}{\MrGoogleT}{LetterGoogle}

Из рекомендательного письма Akvelon:

\ParagraphQuoteByExhibit{%
    <\dots>%
}{\MrAkvelonT}{LetterAkvelon}

Из рекомендательного письма от \MrEditorT:

\ParagraphQuoteByExhibit{%
    Господин Инкин и команда под его руководством в Akvelon <\dots>, и их функционал включает:

    \begin{itemize}

        \item Сворачивание кода.
        Крупные программы бывает тяжело читать.
        Их редактор позволяет сворачивать и раскрывать блоки кода, что значительно повышает удобство для программистов.

        \item Подсветка ошибок.
        Их редактор может отправлять запросы на сервер для проверки кода на ошибки, подсвечивать
        проблемные строки кода и показывать сообщения об ошибках.

        \item Автодополнение.
        Их редактор подсказывает слова по мере того, как пользователь печатает их, и это
        значительно уменьшает количество ошибок.

        \item Поиск.
        Во Flutter нет встроенного кросс-платформенного поиска в текстовых полях, а они сделали его.

    \end{itemize}

    До того, как они выпустили редактор с этим функционалом, редакторы в приложениях Flutter закрывали только базовые потребности программистов,
    оставляя ощущение, что до настоящего профессионального десктоп-редактора далеко.
    В результате программисты часто писали большие фрагменты кода в других редакторах
    и потом вставляли его в приложение, где этот код был им нужен.

    \ul{Функционал, разработанный Алексеем Инкиным, поменял эту парадигму}, теперь больше всего может быть сделано
    непосредственно в приложении на Flutter, и профессиональные \ul{десктопные редакторы нужны реже.
    Это имеет огромное значение для индустрии, поскольку упрощает работу,
    снижает зависимости от десктопа и делает большее количество задач решаемыми чисто на мобильных устройствах}.

    Мы уже \ul{видим это по широкому применению их редактора}.
    Он стал вторым по популярности по статистике репозитория pub.dev.
    Он уже \ul{используется в чатах, онлайн-редакторах конфигурации, крупных онлайн-редакторах и т.п}.%
}{\MrEditorT}{LetterEditor}

Среди приложений, в которых используется flutter\_code\_editor:

\begin{itemize}

    \item Apache Beam Playground, Tour of Beam -- 7100 звёзд \ExhibitRef{BeamGitHubStars}.

    \item network\_proxy\_flutter -- 2700 звёзд, open-source приложение для логирования сетевого трафика
    \ExhibitRef{FceNetworkProxyFlutter}.

    \item Miru -- 900 звёзд, приложение для просмотра видео, комиксов и книг
    \ExhibitRef{FceMiru}.

\end{itemize}

Некоторые отзывы на редактор:

\ParagraphQuoteByExhibit{%
    Я использую flutter\_code\_editor, чтобы пользователи могли редактировать разные скрипты прямо в приложении.
    Я нахожу сворачивание кода очень полезным, особенно на мобильных устройствах с маленьким экраном.
    Это важный фактор в том, почему пользователи любят моё приложение и его популярность растёт.%
}{Автор приложения network\_proxy\_flutter}{FceNetworkProxyFlutterRecommendation}

\ParagraphQuoteByExhibit{%
    Я использую flutter\_code\_editor, разработанный Алексеем Инкиным и Akvelon.

    В нём есть всё, что мне нужно от редактора кода, и оно сразу работает.
    Это позволило мне сфокусироваться на других важных частях моей IDE, например, на работе с LSP-сервером.

    В частности, в редакторе есть анализ кода и возможность подменить анализатор,
    и прекрасная реализация сворачивания кода.
    Сейчас невозможно представить себе IDE без этого.
    Это единственный пакет на Flutter с такими возможностями.
    Без него я бы потратил месяцы, чтобы реализовать это самостоятельно, прежде чем смог бы приступить к основному функционалу своей IDE.%
}{\MrIde}{IdeRecommendation}
