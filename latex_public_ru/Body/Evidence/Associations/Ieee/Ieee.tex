\subsubsection{Институт Инженеров Электрики и Электроники (IEEE), старшее членство}
\label{subsubsec:AssociationsIeee}

\SubSubSubSection{Доказательство, что Алексей Инкин -- член IEEE}

Алексей Инкин -- член IEEE, и это подтверждается его членской карточкой \ExhibitRef{IeeeCard}.


\SubSubSubSection{Доказательство старшего членства Алексея Инкина}

Старший уровень членства Алексея Инкина подтверждается:

\begin{itemize}

    \item Уведомлением от Zuhaina Zakaria, председателя комитета по членству и повышению в 2023 году
    \ExhibitRef{IeeeElevationNotification}.

    \item Поздравительным письмом от Saifur Rahman, президента IEEE в 2023 году \ExhibitRef{IeeePresidentLetter}.

\end{itemize}


\SubSubSubSection{Доказательство, что IEEE -- профессиональная ассоциация в сфере Алексея Инкина -- разработке ПО}

Статья в Wikipedia про IEEE содержит следующие факты:

\ParagraphQuote{%
    Институт Инженеров Электрики и Электроники (IEEE) --
    это профессиональная ассоциация инженеров электрики, инженеров электроники
    и других смежных дисциплин, созданная по классификации 501(c)(3) со штаб-квартирой в городе Нью-Йорк.%
}

\ParagraphQuote{%
    IEEE берёт своё начало в 1884 году.%
}

\ParagraphQuoteByExhibit{%
    В 2021 году в IEEE было больше 400,000 членов из 160 стран.%
}{Статья об IEEE в Wikipedia}{IeeeWikipedia}

Это делает IEEE одной из крупнейших и влиятельнейших профессиональных ассоциаций для инженеров.

Это независимо подтверждается новостью на сайте Samsung,
где говорится, что IEEE это

\ParagraphQuoteByExhibit{%
    крупнейшая в мире техническая профессиональная организация%
}{Новость на сайте Samsung}{Samsung}

Кроме того, сферы интересов IEEE включают
\QuoteByExhibit{\ul{компьютерные науки и информационные технологии}}{Страница с требованиями к старшим членам IEEE}{IeeeSeniorRequirements},
которые включают специализацию Алексея Инкина -- разработку софта.


\SubSubSubSection{Доказательства, что выдающиеся достижения требуются для старшего членства в IEEE}

Сайт IEEE перечисляет требования к старшему членству.
Среди прочего требуется вот что:

\ParagraphQuote{%
    Кандидаты должны быть в профессии минимум 10 лет%
}
\ParagraphQuoteByExhibit{%
    Кандидаты должны продемонстрировать значимую работу на протяжении как минимум 5 из этих лет%
}{Страница с требованиями к старшим членам IEEE}{IeeeSeniorRequirements}

`Значимая работа' в свою очередь определяется так:

\ParagraphQuoteByExhibit{%
    \ul{Значительные рабочие обязанности}, такие как: руководитель команды; супервайзер;
    инженер, отвечающий за программу или проект; инженер или учёный,
    выполняющий исследование \ul{с некоторым подтверждением успешности (публикации)};
    преподаватель, разрабатывающий и проводящий свои курсы \ul{с исследованиями и публикациями} --
    подтверждают значимую работу.
    Следующее тоже является покзателем значимой работы:

    \begin{itemize}

        \item \ul{Значительные} инженерные обязанности или \ul{достижения}

        \item Публикации инженерных или научных статей, книг или изобретений

        \item Техническое руководство важной научной или инженерной работой
        \ul{с доказательствами достижений}

        \item \ul{Признанный вклад} в благополучие научной или инженерной профессии
    \end{itemize}
\dots}{Страница с требованиями к старшим членам IEEE}{IeeeSeniorRequirements}

Это подтверждает, что стаж и важная должность сами по себе не дают право
на старшее членство, а требуются именно достижения.

Также важно, что

\ParagraphQuoteByExhibit{%
    \ul{Старшее членство -- это самый высокий уровень, на который член может подать заявку}.%
}{Главная страница старшего членства в IEEE}{IeeeSeniorHighest}

Политика USCIS напрямую позволяет использовать для этого критерия
уровни членства в ассоциациях, требующие значимый вклад:

\ParagraphQuoteByExhibit{%
    Как возможный пример, базовое членство в международной организации
    для инженеров и технологических профессионалов может не соответствовать требованиям для критерия.
    Но в то же время, если в той же самой организации уровень fellow требует, в частности,
    чтобы у кандидата были достижения, которые, например, внесли \ul{важный вклад}
    в развитие или применение инженерии, науки или технологий,
    и кандидатов оценивает совет экспертов, которые сами на этом же уровне,
    то этот уровень может подходить.

    Другой подходящий пример -- членство на уровне fellow
    в научном сообществе, посвящённом искусственному интеллекту, если членство
    основано на признании внесённого кандидатом \ul{значимого
    продолжительного вклада} в область искусственного интеллекта,
    и совет текущих экспертов на этом же уровне отбирает новых членов.%
}{\PolicyManual}{AssociationsUscisPolicy}


\SubSubSubSection{%
    Доказательство, что заявления страших членов оценивают
    эксперты в своих дисциплинах, признанные на национальном или международном уровне%
}

Оценка международно признанными экспертами началась ещё на этапе сбора рекомендаций
от других старших членов, которые нужно собрать для отправки заявки.
Рекомендатели Алексея Инкина показаны в \ExhibitRef{IeeeReferences}.

Среди рекомендателей был \MrIeeeReferenceOne,
разработчик ОП с особенно выдающейся репутацией.
Он основатель <IEEE Reference 1 Project 1>, компонента для <\dots>,
у которого больше, чем <много> звёзд на GitHub
(см. \ExhibitRef{GitHubStars} о том, почему это действенная метрика популярности),
и который используется в более чем <очень много> других open-source проектов на GitHub \ExhibitRef{IeeeReferenceOneProjectOneGitHub}.
Его авторство подтверждается сопоставлением логина GitHub \Quote{<username>} \ExhibitRef{IeeeReferenceOneGitHub}
с адресом репозитория \ExhibitRef{IeeeReferenceOneProjectOneGitHub}.

\MrIeeeReferenceOne также является автором <IEEE Reference 1 Project 2> -- <\dots>
с более чем <много> звёзд на GitHub \ExhibitRef{IeeeReferenceOneProjectTwoStars}.
Его авторство подтверждается тем, что он старейший и самый активный контрибюьтор
в коде проекта \ExhibitRef{IeeeReferenceOneProjectTwoContributors}.

Страница \MrIeeeReferenceOne в LinkedIn -- \ExhibitRef{IeeeReferenceOneLinkedIn}.

Как сказано на сайте IEEE,
присвоением звания старшего члена занимается комиссия оценщиков при комитете по членству и повышению.

Вот как она формируется:

\ParagraphQuoteByExhibit{%
    Комиссия оценщиков набирается среди старших членов,
    пожизненных старших членов и Fellows в том отделении, где будет проводиться собрание.%
}{Сайт IEEE}{IeeeReviewPanel}

Комиссия, на которой Алексей Инкин получил своё звание старшего члена,
работала под личным руководством доктора Zuhaina Zakaria, как она сама сообщила на LinkedIn \ExhibitRef{ZakariaMeeting}.

Она -- председатель комитета по членству и повышению IEEE в 2023 году \ExhibitRef{IeeeElevationNotification}.

Она также декан Института продолжения образования и
профессор Университета Технологии MARA в Малайзии.

Она

\ParagraphQuoteByExhibit{%
    Опытный профессор с доказанной историей работы в индустрии управления образованием.
    Сильные образовательные профессиональные навыки в математическом моделировании, электронном обучении, Matlab,
    электрических системах и компьютерных науках.%
}{Страница доктора Zuhaina Zakaria в LinkedIn}{ZakariaLinkedIn}

Это подтверждает, что достижения кандидатов в старшие члены оцениваются
экспертами в своих дисциплинах, признанными на национальном или международном уровне.
