\subsubsection{PHPStan}
\label{subsubsec:ContributionsPhpStan}

\SubSubSubSection{Context}

PHP is a popular programming language used by over 19\% of all programmers globally \ExhibitRef{PhpStackOverflow}.

It is a language for website backends where it dominates:

\ParagraphQuoteByExhibit{%
    PHP is used by 76.7\% of all the websites whose server-side programming language we know.%
}{W3Tech}{PhpUsage}

It is used by such large websites as Facebook.com, Wikipedia.org, Vimeo.com, and others \ExhibitRef{PhpWebsites}.

PHP is not shipped with tools to check code for errors before it runs.
This is why external tools have emerged.
One of them is named PHPStan.
It

\ParagraphQuoteByExhibit{%
    \dots finds bugs in your code without writing tests. It's open-source and free.%
}{PHPStan home page}{PhpStanHome}

\mrl has made a major improvement in that tool, which is explained below.


\SubSubSubSection{Proof that \mrl has made an improvement in PHPStan}

The specific pull requests (code contributions) of \mrl are shown in \ExhibitRef{PhpStanPrs}.
The nature of them will be explained below.


\SubSubSubSection{The problem and the nature of \mrls improvement in PHPStan}

The problem and the nature of the \mrls improvement in PHPStan is best described
by \MrPhpOne, a <Title and company>:

\ParagraphQuoteByExhibit{%
    Back in 2020, there was \ul{a long-missing feature} in this tool.
    In simple words, the tool \ul{required redundant comments} in the code
    to correctly detect certain types of errors.
    \ul{These comments were of no other use, did not help the primary functionality of a program,
    and were hard to maintain.
    For that reason, some programmers were never using them,
    exposing their code to a certain class of errors}
    that could only be seen when a program was manually tested.
    My work at the time was affected by this problem as well.

    Technically speaking, that missing feature was the ability to merge and partially override
    the content of doc comments in subclasses.
    Before Alexey's contribution,
    any doc comment in a subclass would replace the one from the base class,
    causing a loss of type information for arguments and return values.
    This was critical because back then a doc comment
    was the only way in PHP to widen an argument type or narrow down a return type,
    and is still the only way to specify the type of elements in an array.
    If a programmer needed to amend any of those types in a subclass,
    they had to duplicate the entire comment from the base class
    for PHPStan to be able to detect type errors.

    Mr. Alexey Inkin has developed and contributed the ability to merge such comments
    while preserving the type information.

    The effects of that contribution were that:

    \begin{enumerate}

        \item \ul{No code duplication is required} for type checks to work when overriding doc comments.

        \item \ul{No false positive errors} are emitted if two comments fall out of sync by mistake.

        \item The behavior of the tool now matches what programmers intuitively expect,
        thus they \ul{spend less time learning special behavior}.

    \end{enumerate}

    As I already mentioned, PHPStan is \ul{the most popular} static code analyzer for PHP
    with \ul{over a hundred million downloads} since the contribution of Mr. Alexey Inkin.
    \ul{It's safe to say that this feature is widely used in the industry in companies of all sizes}.
}{\MrPhpOneT}{LetterPhpOne}

Mr. \MrPhpOne is <title in company>, see \ExhibitRef{PhpOneCompanyRole}
and his LinkedIn page \ExhibitRef{PhpOneLinkedIn}.

The <\dots> role of <the company> for the PHP language and community is described
on the website of <the company>:

\ParagraphQuoteByExhibit{%
    <\dots>%
}{<Company website>}{PhpOneCompanyAbout}

The problem was not only that excessive comments were required for the analyzer to work properly.
As a <title> puts it,

\ParagraphQuoteByExhibit{%
    To preserve type information, programmers had to fully copy the doc block
    from the original method and then edit some types.

    This was inconvenient and hard to maintain, so \ul{programmers mostly never did that.
    Instead, they were not using this check to avoid false reports
    when their code relied on more specific types in a subclass.
    Skipping a check results in possible errors.
    Even large projects with over a million lines of code were suffering from that}.

    Mr. Alexey Inkin resolved that.
    He implemented merging of the doc blocks when overriding a method.
    A ‘return’ and ‘param’ tags in an overridden method now take over, and ‘throws’ tags add up.
    This removed the need for redundant comments in code
    and matches the intuitive understanding of how things should work.
    \ul{This allowed PHPStan to be adopted in large projects} without silencing type checks,
    which resulted in \ul{improved code quality and stability of projects in the industry}.
    The tool has seen a \ul{great increase of usage after May 2020 when this was implemented}.%
}{\MrPhpTwoT}{PhpTwoRecommendation}

The position of \MrPhpTwo is confirmed by his LinkedIn page \ExhibitRef{PhpTwoLinkedIn}
and an announcement from <the company> \ExhibitRef{PhpTwoCompanyRole}.

`<title>' is an important position
because <of what is explained in> \ExhibitRef{PhpTwoCompanyAbout}.

The problem in PHPStan was so critical and long-standing
that it was reported 11 times independently although programmers normally check
if an issue was already reported before.
See \ExhibitRef{PhpStanIssues} for the list of all issues
fixed by this contribution.
Note that in the end of all of them the creator of PHPStan attributes

\ParagraphQuoteByExhibit{%
    Awesome work!%
}{\MrMirtesT}{PhpStanIssues}

to \mrl personally.

The fact that \MrMirtes is the founder of PHPStan
is confirmed on his LinkedIn page \ExhibitRef{MirtesLinkedIn}.


\SubSubSubSection{Proof of the mass adoption of PHPStan}

Packages in PHP are traditionally downloaded from the website called Packagist \ExhibitRef{ComposerWikipedia},
and it tracks the download statistics for individual packages.
It shows that PHPStan has been downloaded over 131 million times
since the contribution of \mrl \ExhibitRef{PhpStanPackagist}.

Another metric is GitHub stars.
PHP itself has just over 36 thousand stars while PHPStan has over 12 thousands \ExhibitRef{PhpStanStars}.
If a supportive tool has just 3 times less stars than the language it supports,
it means the tool is very popular and important.
See \ExhibitRef{GitHubStars} on why GitHub stars is a valid popularity metric.
