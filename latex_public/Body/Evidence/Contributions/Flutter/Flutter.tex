\subsubsection{Flutter Code Editor}
\label{subsubsec:ContributionsFlutter}

\SubSubSubSection{Context}

Flutter is a technology developed by Google for cross-platform development of web, mobile, and computer applications.
According to Google's own account, it helped them reduce the engineering efforts by 70\%
in one of their applications compared to separate development for Android and iOS \ExhibitRef{FlutterGooglePay}.

`Flutter Code Editor' is a software product developed at Akvelon, led by \mrl.
It is a code editor that can be used as part of any application made with Flutter.
It has unique combination of features and supported platforms, as will be shown below,
thus it is an original scientific contribution in the field.

For better context and the visual demo of the editor, see \ExhibitRef{FceDemo}.

\SubSubSubSection{Proof that \mrl is the primary author of the editor}

The leading role of \mrl is shown in the letter from \MrAkvelonT:

\ParagraphQuoteByExhibit{%
    <\dots>%
}{\MrAkvelonT}{LetterAkvelon}

On GitHub, it can be independently seen that \mrl has made almost 5 times more commits
to the editor source code than the next largest contributor \ExhibitRef{FceGitHubContributors}.

Finally, \mrl is listed as the first of the team members in the product announcement
\ExhibitRef{AkvelonFceAnnouncement}.


\SubSubSubSection{Proof that the product is the second most popular universal code editor in Flutter}

The official package repository for Dart and Flutter packages is `https://pub.dev'.
It shows the popularity metric for packages but it has no categories of packages.

To specifically select code editors to compare their popularity,
the website FlutterGems.dev can be used.
As explained in their Twitter account, it is

\ParagraphQuoteByExhibit{%
    a curated package guide which functionally categorizes 5.5k+ most useful and popular @dart\_lang [Dart]
    \& @FlutterDev [Flutter] packages.%
}{Flutter Gems Twitter}{FlutterGemsTwitter}

For context, Dart is the programming language which is used in Flutter technology,
they come together often in this petition.

FlutterGems.dev has a category for code editors and highlighters, which contains 12 packages
\ExhibitRef{FlutterGemsEditors}.
There are 7 packages with good and average maintenance status.
Of them, the following should be excluded:

\begin{itemize}

    \item flutter\_highlighter and highlighter are not editors,
    they only highlight static uneditable code.

    \item json\_editor is a specialized editor for only a single language, JSON \ExhibitRef{JsonWikipedia},
    while universal code editors are expected to support over a hundred programming languages.

\end{itemize}

If sorted by the number of likes in the original pub.dev repository, the remaining packages are:

\begin{center}
    \begin{tabular}{|l|l|r|r|}
        \hline
        & Name & Likes & Popularity\\
        \hline
        1 & code\_text\_field & 135 & 93\%\\
        2 & code\_editor & 94 & 83\%\\
        3 & flutter\_code\_editor & 80 & 89\%\\
        4 & codemirror & 15 & 83\%\\
        \hline
    \end{tabular}
\end{center}

In this table, the popularity metric is taken from each package's pub.dev page,
the original source of truth for all statistics \ExhibitRef{CodeEditorPopularity}.

As explained on a help page at pub.dev, likes are what developers manually award
while the popularity is

\ParagraphQuoteByExhibit{%
    a percentile from 100\% (among the top 1\% most used packages)
    to 0\% (the least used package).%
}{Pub.dev help page}{PubScoring}

So the popularity is the proper metric and not likes.
This makes flutter\_code\_editor the second most popular universal code editor in Flutter.

Note that `codemirror', the 4th package in the list, is made by Google
as indicated on its pub.dev page under `Publisher' section \ExhibitRef{CodeEditorPopularity}.
\mrls product in this category is more popular and more widely used than the competing product from Google itself.

Note: The popularity percentiles of 93\% and 89\% should not be considered low by themselves.
Code editors are niche packages to make applications for programmers.
They are expected to be less popular than some fancy buttons, lists, timers, etc.,
which are used in mainstream applications and have the popularity of 98--100\% on pub.dev.


\SubSubSubSection{Proof of the major significance of the product}

The supporting letter from Google reads:

\ParagraphQuoteByExhibit{%
    <\dots>%
}{\MrGoogleT}{LetterGoogle}

The supporting letter from Akvelon reads:

\ParagraphQuoteByExhibit{%
    <\dots>%
}{\MrAkvelonT}{LetterAkvelon}

The letter from Mr. \MrEditorT reads:

\ParagraphQuoteByExhibit{%
    Mr. Inkin and the team he led while working for Akvelon <\dots> their most prominent features include:

    \begin{itemize}

        \item Code folding.
        Large programs can get hard to read.
        Their editor can collapse and expand code sections, and it greatly improves programmer's experience.

        \item Error highlighting.
        Their editor can connect to servers to check code for errors and show
        problematic lines and report specific problems in code.

        \item Autocomplete.
        Their editor suggests words as a programmer types them, and it
        significantly reduces programming errors.

        \item Search.
        Flutter does not have a cross-platform built-in search in text fields, and they made it.

    \end{itemize}

    Before those features, an editor in a Flutter app could only satisfy basic programmer's needs
    while leaving the feeling that it's still far from professional development tools available on computers.
    As a result, people would often write larger pieces of code in professional desktop
    editors and then copy and paste it into a Flutter app to use it there.

    \ul{The features developed by Mr. Inkin led to a paradigm shift} where much more work can be done
    in a Flutter app directly, and professional \ul{desktop editors are less often needed.
    This is huge for the industry because it streamlines work,
    decreases dependency on desktop computers and makes more tasks achievable on just mobile devices}.

    This is already \ul{seen by the adoption of their editor}.
    It became the second most popular one as measured by the package repository at pub.dev.
    It is already \ul{used in chats, online configuration editors, larger-scale online editors, and more}.%
}{\MrEditorT}{LetterEditor}

Applications that use flutter\_code\_editor include:

\begin{itemize}

    \item Apache Beam Playground, Tour of Beam -- 7100 stars \ExhibitRef{BeamGitHubStars}.

    \item network\_proxy\_flutter -- 2700 stars, an open-source free packet capture software
    \ExhibitRef{FceNetworkProxyFlutter}.

    \item Miru -- 900 stars, a viewer for video, comics, and novels
    \ExhibitRef{FceMiru}.

\end{itemize}

Reviews of the editor include:

\ParagraphQuoteByExhibit{%
    I use flutter\_code\_editor to allow users to edit various scripts right within the app.
    I find the code folding feature very valuable especially on small screens of mobile devices.
    It is an important factor to why the users love the app and its popularity grows.%
}{The author of network\_proxy\_flutter application}{FceNetworkProxyFlutterRecommendation}

\ParagraphQuoteByExhibit{%
    I use flutter\_code\_editor developed by Alexey Inkin and Akvelon, Inc.

    It comes with all the things that I need in a code editor out of the box.
    That has helped me focus on the other critical aspects of the IDE like the LSP server.

    Specifically, it has an out of the box solution for code analysis and a way to override its analyser,
    and great implementation of code blocks folding.
    No IDE could possibly do without that nowadays.
    This is the only package in Flutter having those features.
    Without it, I would have to spend months implementing it myself before I could get to the IDE functionality.%
}{\MrIde}{IdeRecommendation}
